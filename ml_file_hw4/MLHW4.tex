\documentclass[12pt]{article}
\usepackage{graphicx}
\usepackage{amsmath}

\begin{document}
CSCI-4100 Assignment 4\\
Yichuan Wang \\
RIN:661414395\\\\

EXERCISES\\\\
2.4
(a)We can construct a input matrix as following:
\[
\begin{bmatrix}
    1       & 0 & 0 & \dots & 0 \\
    1       & 1 & 0 & \dots & 0 \\
    1       & 0 & 1 & \dots & 0 \\
    \hdotsfor{5} \\
    1       & 0 & 0 & \dots & 1
\end{bmatrix}
\]
Since the matrix is full rank, it is invertible.\\
Since we have $Xw = y$ and X is invertible, we have $w=X^{-1}y$ for all y where y is a d+1 length column of +1 and -1.\\ 
Since we can calculate w for each y using equation $w=X^{-1}y$, there must be at least one w for each y, and therefore these $d+1$ data points can be shattered.\\ 
(b)Due to the fact that any $d+2$ vectors of length $d+1$ has to be linearly dependent, there must be one vector $x_{d+2}$ such that $x_{d+2}=c_0x_0+c_1x_1+c_2x_2+...+c_{d+1}x_{d+1}$. Suppose there is a dichotomy $A$ of first $d+1$ independent vectors such that $w^Tc_nx_n<0$ for all coefficients $c$, then $sign(w^Tx_{d+2})$ must be -1. In this case the perceptron can only implement dichotomy $[A,-1]$; it cannot implement $[A,+1]$.\\  

PROBLEMS\\\\
2.3\\
(a)\\
$m_H(1)=2=2^1$ 	A single point can be classified as +1 or -1.\\
$m_H(2)=4=2^2$	2 points can have 4 dichotomies. They can be both 0 and 1, and have both {0,1} and {1,0} configuration since the ray can be positive or negative when placed in between these two points. \\
$m_H(3)=6<2^3$ 	Since the data points are on a 1D array, the middle point must be the same as one of the side points, and therefore {0,1,0} and {1,0,1} cannot be achieved.\\
The VC dimension for positive or negative ray is 2.\\
(b)\\
According to Page 44 of LFD text book, a positive interval has a $m_H(N)=\frac{1}{2}N^2+\frac{1}{2}N+1$. If $H$ contains both positive and negative intervals, it would first cover all the results for positive intervals. It then should cover all the negative intervals. Since the negative interval that covers $x_1$ or $x_n$ has its equivalence case in positive intervals, we just need to consider ${N+1-2}\choose 2$ while setting both end points to be +1. There is no need to have +1 for negative interval since that is also a case covered in positive interval. In this case the total result would be ${{N+1}\choose 2} + 1 + {{N-1}\choose 2} = N^2-N+2$\\
$m_H(1)=2=2^1$\\
$m_H(2)=4=2^2$\\
$m_H(3)=8=2^3$\\
$m_H(4)=14<2^4$\\
The VC dimension is 3\\
(c)This case can be transformed into positive intervals. Each data point will be assigned a real number $r$ based on their feature values $x_1...x_n$, and the concentric sphere is simply a positive interval among these real number values. The VC dimension in this case is 2.\\

2.8. There are only two possible cases for growth function $m_H(N)$: Power of 2 and polynomial. Also since there is theorem: "if $m_H(k)<2^k$ some value $k$, then for all N, $m(N)<N^{k-1}+1$ ". In this case $1+N+\frac{N(N-1)(N-2)}{6}$ cannot be a answer: if we try $N=2$ we get a result of 3, so we can conclude $k=2$. However, it doesn't satisfy $m(N)<N^{k-1}+1$ for $N=2$ ($3=3$).\\
The possible growth function are:\\
$1+N$  This can be a positive positive ray\\
$1+N+\frac{N(N-1)}{2}$  This can be a positive interval\\
$2^N$  $d_{VC}=\infty$\\

2.10\\
The dichotomy of 2N points can be seen as a combination of two dichotomies of N points. The maximum number of combinations of two dichotomies of N points is at most all possible combinations of these two dichotomies, which is $m_H(N)^2$. So $m_H(2N) \leq m_H(N)^2$. \\
Generalization bound: $E_{out}(g)\leq E_{in}(g)+\sqrt{\frac{8}{N}ln(\frac{4m_H(N)^2}{\sigma})}$\\

2.12\\
Iterative calculation starting with $N=1000$\\
$$N\geq \frac{8}{0.05^2}ln(\frac{4(2\times 1000)^{10}+1}{0.05})=257251$$
$$N\geq \frac{8}{0.05^2}ln(\frac{4(2\times 257251)^{10}+1}{0.05})=434853$$
$$N\geq \frac{8}{0.05^2}ln(\frac{4(2\times 434853)^{10}+1}{0.05})=451652$$
$$N\geq \frac{8}{0.05^2}ln(\frac{4(2\times 451652)^{10}+1}{0.05})=452864$$
$$N\geq \frac{8}{0.05^2}ln(\frac{4(2\times 452864)^{10}+1}{0.05})=452950$$
$$N\geq \frac{8}{0.05^2}ln(\frac{4(2\times 452950)^{10}+1}{0.05})=452956$$
$$N\geq \frac{8}{0.05^2}ln(\frac{4(2\times 452956)^{10}+1}{0.05})=452957$$
$$N\geq \frac{8}{0.05^2}ln(\frac{4(2\times 452957)^{10}+1}{0.05})=452957$$

Answer: $N\geq452957$\\

\end{document}